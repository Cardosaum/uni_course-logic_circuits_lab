%%%%%%%%%%%%%%%%%%%%%%%%%%%%%%%%%%%%%%%%%%%%%%%%%%%%%%%%
% Este é um documento que servirá de modelo para
% os relatórios feitos na disciplina Laboratório de Circuitos Lógicos
% 2020-2
%%%%%%%%%%%%%%%%%%%%%%%%%%%%%%%%%%%%%%%%%%%%%%%%%%%%%%%%%

%%%%%%%%%%%%%%%%%%%%%%%%%%%%%%%%%%%%%%%%%%%%%%%%%%%%%%%%%
% Use os diferentes diretórios para colocar os relatórios de cada experimento, deste modo vc consegue manter um histórico e todo material organizado em apenas um local.
% Lembre-se de mudar o Main Document no Menu!!!

\documentclass[12pt]{article}

\usepackage{sbc-template}
\usepackage[brazil,american]{babel}
\usepackage[utf8]{inputenc}

\usepackage{graphicx}
\usepackage{url}
\usepackage{float}
\usepackage{listings}
\usepackage{color}
\usepackage{todonotes}
\usepackage{algorithmic}
\usepackage{algorithm}
\usepackage{hyperref}
\usepackage{amsmath}
\usepackage{graphicx}
\usepackage{array}
\usepackage[shortlabels]{enumitem}

\sloppy


\title{Experimento 5\\
Circuitos Combinacionais: Codificador e Decodificador}

\author{Matheus Cardoso de Souza, 202033507\\
        Ualiton Ventura da Silva, 202033580\\
        Grupo G42
}

%%%% LEMBRE-SE DE MUDAR O GRUPO NA LINHA ABAIXO!!!!! %%%%%%
\address{Dep. Ciência da Computação -- Universidade de Brasília (UnB)\\
  CIC0231 - Laboratório de Circuitos Lógicos
  \email{matheus-cardoso.mc@aluno.unb.br, 202033580@aluno.unb.br}
}

\begin{document}
\maketitle

\selectlanguage{american}
 \begin{abstract}
   TODO
 \end{abstract}
\selectlanguage{brazil}

 \begin{resumo}
   TODO
 \end{resumo}


\section{Introdução}
\label{sec:Introducao}

% Escreva com suas palavras o que vai ser trabalhado no experimento. Aqui temos um exemplo de como citar a bibliografia consultada \cite{boulic:91} \cite{smith:99}.

TODO

\subsection{Objetivos}
\label{sec:Objetivos}

TODO

\subsection{Materiais}
\label{sec:Materiais}
Em função da natureza do ensino a distância, os presentes experimentos não foram
realizados usando-se materiais e equipamentos físicos, mas sim emulados por meio
do \href{https://www.digitalelectronicsdeeds.com/deeds.html}{Deeds}.

A seguir estão enumerados os materiais utilizados:
\begin{itemize}
        TODO
\end{itemize}

\section{Procedimentos}
\label{sec:Procedimentos}
\setcounter{subsection}{-1}

Passaremos a apresentar os experimentos requeridos.

\subsection{TODO}\label{sec:TODO}

TODO

\section{Análise dos Resultados}
\label{sec:Resultados}

TODO

\section{Conclusão}
\label{sec:Conclusao}

TODO

\nocite{*}
\bibliographystyle{sbc}
\bibliography{relatorio}  %Aqui é a definição do arquivo .bib a ser usado pelas referências


\newpage
% Colocar aqui apenas as respostas dos itens da Auto-Avaliação
\section*{Auto-Avaliação}

Respostas:

TODO
% \begin{table}[H]
%      \begin{tabular}{|c|c|} \hline
%      \textbf{A} & \textbf{B}\\
%      \hline
%      1 & c \\ \hline
%      2 & a \\ \hline
%      3 & d \\ \hline
%      4 & a \\ \hline
%      5 & a \\ \hline
%      \end{tabular}
% \end{table}


\end{document}
