%%%%%%%%%%%%%%%%%%%%%%%%%%%%%%%%%%%%%%%%%%%%%%%%%%%%%%%%
% Este é um documento que servirá de modelo para
% os relatórios feitos na disciplina Laboratório de Circuitos Lógicos
% 2020-2
%%%%%%%%%%%%%%%%%%%%%%%%%%%%%%%%%%%%%%%%%%%%%%%%%%%%%%%%%

%%%%%%%%%%%%%%%%%%%%%%%%%%%%%%%%%%%%%%%%%%%%%%%%%%%%%%%%%
% Use os diferentes diretórios para colocar os relatórios de cada experimento, deste modo vc consegue manter um histórico e todo material organizado em apenas um local.
% Lembre-se de mudar o Main Document no Menu!!!

\documentclass[12pt]{article}

\usepackage{sbc-template}
\usepackage[brazil,american]{babel}
\usepackage[utf8]{inputenc}

\usepackage{graphicx}
\usepackage{url}
\usepackage{float}
\usepackage{listings}
\usepackage{color}
\usepackage{todonotes}
\usepackage{algorithmic}
\usepackage{algorithm}
\usepackage{hyperref}
\usepackage{amsmath}
\usepackage{graphicx}
\usepackage{array}

\sloppy


\title{Experimento 1\\
Portas Lógicas AND, OR e NOT}

\author{Matheus Cardoso de Souza, 202033507\\
        Ualiton Ventura da Silva, 202033580\\
        Grupo G42
}

%%%% LEMBRE-SE DE MUDAR O GRUPO NA LINHA ABAIXO!!!!! %%%%%%
\address{Dep. Ciência da Computação -- Universidade de Brasília (UnB)\\
  CIC0231 - Laboratório de Circuitos Lógicos
  \email{matheus-cardoso.mc@aluno.unb.br, 202033580@aluno.unb.br}
}

\begin{document}
\maketitle

\selectlanguage{american}
 \begin{abstract}
   This report corresponds to the Experiment 1 on ``Logical Gates \textbf{NAND}, \textbf{NOR} and \textbf{XOR}''.
 \end{abstract}
\selectlanguage{brazil}

 \begin{resumo}
  O presente relatório corresponde ao Experimento 1 sobre ``Portas Lógicas \textbf{NAND}, \textbf{NOR} e \textbf{XOR}''.
 \end{resumo}


\section{Introdução}
\label{sec:Introducao}

% Escreva com suas palavras o que vai ser trabalhado no experimento. Aqui temos um exemplo de como citar a bibliografia consultada \cite{boulic:91} \cite{smith:99}.

Neste experimento temos como objetivo abordar conceitos referentes ao uso de portas lógicas que conceitualmente derivam de portas primitivas como \textbf{AND}, \textbf{OR} e \textbf{NOT}.

\subsection{Objetivos}
\label{sec:Objetivos}

Os experimentos realizados visam abordar assuntos como a universalidade de portas, assim como o comportamento ocorrido no encadeamento de portas lógicas. Também tem como objetivo descrever como circuitos lógicos \textbf{XOR} de N entradas se comportam de maneira generalizada.

\subsection{Materiais}
\label{sec:Materiais}
Em função da natureza do ensino a distância, os presentes experimentos não foram
realizados usando-se materiais e equipamentos físicos, mas sim emulados por meio
da simulador online \href{https://www.tinkercad.com/}{Tinkercad}.

A seguir estão enumerados os materiais simulados:
\begin{itemize}
    \item Painel Digital
    \item \textit{Protoboard}
    \item Fios
    \item Seletores de estado lógico
    \item LEDs
    \item Resistores
    \item Multímetros
    \item Portas Lógicas \textbf{NAND}, \textbf{NOR} e \textbf{XOR}
\end{itemize}

\section{Procedimentos}
\label{sec:Procedimentos}

Passaremos a apresentar os experimentos requeridos.

\subsection{Implementação de uma porta \textbf{NAND} de 3 entradas}
\label{sec:led_chave}

O atual experimento visa implementar uma porta \textbf{NAND} com três entradas com o uso apenas de portas \textbf{NAND}, sendo também requisitado a medição dos valores lógicos em pontos específicos do circuito.\\
Utilizou-se 3 circuitos integrados NAND, sendo este o CI 74HC00, e utilizando duas entradas(entrada 1A e 1B) de cada um dos mesmos, assim como suas saídas (saída 1), para a verificação dos resultados utilizou-se os leds L0, L1 e L2, para a variação dos valores lógicos usam-se as chaves A,B e C.\\

Abaixo sua implementação:
\begin{figure}[H]
    \centering
    \includegraphics[width=.9\textwidth]{Exp02/Experimento_2.1.png}
    \caption{Circuito \textbf{NAND} de 3 entradas}
    \label{fig:Esquema_Experimento2_2.3}
\end{figure}

Para conferir o vídeo deste experimento, acesse o seguinte link:
\href{https://youtu.be/lwS4AKuvgps}{https://youtu.be/lwS4AKuvgps}.

Abaixo temos a tabela dos valores lógicos obtidos ao realizar o experimento:

\begin{table}[H]
    \centering
    \caption{Tabela para a porta lógica \textbf{NAND} 3 entradas}
    \begin{tabular}{|c|c|c|c|c|c|}
     \hline
    \multicolumn{2}{|c|}{Entradas} & \multicolumn{4}{|c|}{Saídas} \\
    \hline
    \textbf{C} & \textbf{B} & \textbf{A} & $\textbf{L3}=\overline{\rm A.B}$ & $\textbf{L2}=\text{A.B}$ & $\textbf{L1}=\overline{\rm A.B.C}$ \\
    \hline
    0 & 0  & 0 & \(1\) & \(0\) & \(1\) \\
    \hline
    0 & 0  & 1 & \(1\) & \(0\) & \(1\) \\
    \hline
    0 & 1  & 0 & \(1\) & \(0\) & \(1\) \\
    \hline
    0 & 1  & 1 & \(0\) & \(1\) & \(1\) \\
    \hline
    1 & 0  & 0 & \(1\) & \(0\) & \(1\) \\
    \hline
    1 & 0  & 1 & \(1\) & \(0\) & \(1\) \\
    \hline
    1 & 1  & 0 & \(1\) & \(0\) & \(1\) \\
    \hline
    1 & 1  & 1 & \(0\) & \(1\) & \(0\) \\
    \hline
    \end{tabular}\label{tab:tabela_and}
\end{table}

É possível concluir através da tabela e vídeo que de fato descreveu uma porta \textbf{NAND} de três entradas, sendo esta equivalente a $\overline{\rm A.B.C}$.
\\[2em]

\subsection{Implementação da função \textbf{XOR} usando portas \textbf{NAND}.}\label{sec:and_e_or}

Sendo portas \textbf{NAND} universais, são capazes de descrever não somente \textbf{NAND} de três entradas como qualquer outro tipo, assim, este experimento realiza a implementação de uma porta \textbf{XOR} utilizando \textbf{NAND}.\\
Utilizou-se 4 circuitos integrados NAND, sendo CI 74HC00, para cada circuito integrado utilizou-se portas de 2 entradas(entrada 1A e 1B) , como saídas, utiliza-se a saída das respectivas portas(Saída 1), para o resultado, dispomos do led L0.\\

Abaixo a implementação do circuito:
\begin{figure}[H]
    \centering
    \includegraphics[width=.9\textwidth]{Exp02/Experimento_2.2.png}
    \caption{Circuito \textbf{XOR} com \textbf{NAND}}\label{fig:Esquema_Experimento2_2.2}
\end{figure}
Para conferir o vídeo deste experimento, acesse o seguinte link:
\href{https://youtu.be/QhaJeFVEhhw}{https://youtu.be/QhaJeFVEhhw}.

\begin{table}[H]
    \centering
    \caption{Tabela para a porta lógica \textbf{AND}}
    \begin{tabular}{|c|c|c|}
    \hline
        \multicolumn{2}{|c|}{Entradas} & \multicolumn{1}{c|}{Saídas}\\
    \hline
    \textbf{B} & \textbf{A} & \textbf{L0}=A$\oplus$B \\
    \hline
    0  & 0 & \(0\) \\
    \hline
    0  & 1 & \(1\) \\
    \hline
    1  & 0 & \(1\) \\
    \hline
    1  & 1 & \(0\) \\
    \hline
    \end{tabular}\label{tab:tabela_and}
\end{table}

Portanto, verifica-se através tanto do vídeo quanto tabela que de fato foi possível descrever a porta lógica \textbf{XOR} através de portas \textbf{NAND}.\\[2em]

\subsection{Implementação de uma porta \textbf{XOR} de 4 entradas usando portas \textbf{XOR} de 2 entradas}
O atual experimento realiza a implementação de uma porta \textbf{XOR} de quatro entradas com o uso de portas \textbf{XOR} de duas entradas, assim, esquematicamente temos que poderá ser descrito como:

\begin{figure}[H]
    \centering
    \includegraphics[width=.9\textwidth]{Exp02/Esquema_2.3.png}
    \caption{Esquema de uma porta \textbf{XOR} de 4 entradas}\label{fig:Esquema_Experimento2_2.3}
\end{figure}

Expressando através do Thinkercad e com o uso do circuito 74HC86 teremos:

\begin{figure}[H]
    \centering
    \includegraphics[width=.9\textwidth]{Exp02/Experimento_2.3.png}
    \caption{Circuito \textbf{XOR} de 4 entradas}\label{fig:Esquema_Experimento2_2.3}
\end{figure}

\\
Para conferir o vídeo deste experimento, acesse o seguinte link:
\href{https://youtu.be/CHNdltWfVFs}{https://youtu.be/CHNdltWfVFs}.
\\
Assim, sua tabela verdade será:

\begin{table}[H]
    \centering
    \caption{Tabela para a porta lógica \textbf{XOR 4 Entradas}}
    \begin{tabular}{|c|c|c|c|c|}
    \hline
        \multicolumn{4}{|c|}{Entradas} & \multicolumn{1}{|c|}{Saídas}\\
    \hline
    \textbf{D} & \textbf{C} & \textbf{B} & \textbf{A} & L0\\
    \hline
    0 & 0 & 0 & 0 & \(0\) \\
    \hline
    0 & 0 & 0 & 1 & \(1\) \\
    \hline
    0 & 0 & 1 & 0 & \(1\) \\
    \hline
    0 & 0 & 1 & 1 & \(0\) \\
    \hline
    0 & 1 & 0 & 0 & \(1\) \\
    \hline
    0 & 1 & 0 & 1 & \(0\) \\
    \hline
    0 & 1 & 1 & 0 & \(0\) \\
    \hline
    0 & 1 & 1 & 1 & \(1\) \\
    \hline
    1 & 0 & 0 & 0 & \(0\) \\
    \hline
    1 & 0 & 0 & 1 & \(0\) \\
    \hline
    1 & 0 & 1 & 0 & \(0\) \\
    \hline
    1 & 0 & 1 & 1 & \(1\) \\
    \hline
    1 & 1 & 0 & 0 & \(0\) \\
    \hline
    1 & 1 & 0 & 1 & \(1\) \\
    \hline
    1 & 1 & 1 & 0 & \(1\) \\
    \hline
    1 & 1 & 1 & 1 & \(0\) \\
    \hline
    \end{tabular}
    \label{tab:tabela_and}
\end{table}

Através da tabela e vídeo apresentado é possível notar que uma porta \textbf{XOR} de 4 entradas possui valor lógico 1 quando o número de bits é ímpar, caso contrário será 0, este comportamento não é constante, mas de maneira geral tem-se que para uma porta \textbf{XOR} de N entradas, se N é par, então quando o seu número de bits de entrada que assumem valor lógico 1 for ímpar, seu resultado será 1, contrariamente temos que se N é ímpar, então quando o número de bits de entrada assumir valor lógico 1 for par, assim teremos valor lógico 1.

\section{Análise dos Resultados}
\label{sec:Resultados}

Através dos experimentos realizados foi possível comprovar a universalidade de portas \textbf{NAND} e também como portas \textbf{XOR} de N entradas possuem um regra geral que é capaz de descrever suas saídas.

\section{Conclusão}
\label{sec:Conclusao}

Utilizando conceitos referentes a universalidade de portas temos uma ferramenta que possibilita maior versatilidade no dia a dia, a princípio poderia parecer ilógico do porquê é útil este conceito, contudo, deve-se notar que nem sempre possuímos todas as portas desejadas disponíveis, assim, ter recursos alternativos para lidar com situações como estas é importante.
Referente ao uso de portas \textbf{XOR} temos que seus conceitos de paridade são muito propícios em análise de bits, exemplo a ser citado é através da transmissão de dados, e um exemplo de código que utiliza de tal característica é o código de Hamming, que através da análise de paridade de bits é capaz de realizar a detecção de erros em transmissão de dados, criando então uma camada de segurança.

\nocite{*}
\bibliographystyle{sbc}
\bibliography{relatorio}  %Aqui é a definição do arquivo .bib a ser usado pelas referências


\newpage
% Colocar aqui apenas as respostas dos itens da Auto-Avaliação
\section*{Auto-Avaliação}

Respostas:

\begin{table}[H]
    \begin{tabular}{|c|c|} \hline
    \textbf{A} & \textbf{B}\\
    \hline
    1 & c \\ \hline
    2 & a \\ \hline
    3 & d \\ \hline
    4 & c \\ \hline
    5 & b \\ \hline
    \end{tabular}
\end{table}


\end{document}
