%%%%%%%%%%%%%%%%%%%%%%%%%%%%%%%%%%%%%%%%%%%%%%%%%%%%%%%%
% Este é um documento que servirá de modelo para
% os relatórios feitos na disciplina Laboratório de Circuitos Lógicos
% 2020-2
%%%%%%%%%%%%%%%%%%%%%%%%%%%%%%%%%%%%%%%%%%%%%%%%%%%%%%%%%

%%%%%%%%%%%%%%%%%%%%%%%%%%%%%%%%%%%%%%%%%%%%%%%%%%%%%%%%%
% Use os diferentes diretórios para colocar os relatórios de cada experimento, deste modo vc consegue manter um histórico e todo material organizado em apenas um local.
% Lembre-se de mudar o Main Document no Menu!!!

\documentclass[12pt]{article}

\usepackage{sbc-template}
\usepackage[brazil,american]{babel}
\usepackage[utf8]{inputenc}

\usepackage{graphicx}
\usepackage{url}
\usepackage{float}
\usepackage{listings}
\usepackage{color}
\usepackage{todonotes}
\usepackage{algorithmic}
\usepackage{algorithm}
\usepackage{hyperref}
\usepackage{amsmath}
\usepackage{graphicx}
\usepackage{array}
\usepackage{mwe}
\usepackage[shortlabels]{enumitem}

\usepackage{xcolor}
\usepackage{listings}
\usepackage[electronic]{ifsym}
\definecolor{vgreen}{RGB}{104,180,104}
\definecolor{vblue}{RGB}{49,49,255}
\definecolor{vorange}{RGB}{255,143,102}

\lstdefinestyle{verilog-style}
{
    language=Verilog,
    basicstyle=\small\ttfamily,
    keywordstyle=\color{vblue},
    identifierstyle=\color{black},
    commentstyle=\color{vgreen},
    numbers=left,
    numberstyle=\tiny\color{black},
    numbersep=10pt,
    tabsize=8,
    moredelim=*[s][\colorIndex]{[}{]},
    literate=*{:}{:}1
}

\makeatletter
\newcommand*\@lbracket{[}
\newcommand*\@rbracket{]}
\newcommand*\@colon{:}
\newcommand*\colorIndex{%
    \edef\@temp{\the\lst@token}%
    \ifx\@temp\@lbracket \color{black}%
    \else\ifx\@temp\@rbracket \color{black}%
    \else\ifx\@temp\@colon \color{black}%
    \else \color{vorange}%
    \fi\fi\fi
}
\makeatother

\usepackage{trace}

\sloppy


\title{Projeto Final\\
ZeptoProcessador-III de 16 Bits}

\author{Matheus Cardoso de Souza, 202033507\\
        Ualiton Ventura da Silva, 202033580\\
        Grupo G42
}

%%%% LEMBRE-SE DE MUDAR O GRUPO NA LINHA ABAIXO!!!!! %%%%%%
\address{Dep. Ciência da Computação -- Universidade de Brasília (UnB)\\
  CIC0231 - Laboratório de Circuitos Lógicos
  \email{matheus-cardoso.mc@aluno.unb.br, 202033580@aluno.unb.br}
}

\begin{document}
\maketitle

\selectlanguage{american}
 \begin{abstract}
   \textbf{TODO}
 \end{abstract}

\selectlanguage{brazil}
 \begin{resumo}
   \textbf{TODO}
 \end{resumo}


\section{Introdução}\label{sec:Introducao}

\textbf{TODO}

\subsection{Objetivos}\label{sec:Objetivos}

\textbf{TODO}

\subsection{Materiais}\label{sec:Materiais}

Em função da natureza do ensino a distância, os presentes experimentos não foram
realizados usando-se materiais e equipamentos físicos, mas sim emulados por meio
do software
\href{https://www.digitalelectronicsdeeds.com/downloads.html}{Deeds}.

A seguir estão enumerados os materiais utilizados:
\begin{itemize}
    \item Software Deeds
    \item Portas lógicas
    \begin{itemize}
      \item \emph{NAND}s de $2$ entradas
      \item \emph{NOR}s de $2$ entradas
      \item \emph{Flip-flops JK-net}s
      \item \emph{Display de saída de 8 Segmentos}
      \item \emph{Display de saída de 1 bit}
    \end{itemize}
    \item \emph{Clocks}
\end{itemize}

\section{Procedimentos}\label{sec:Procedimentos}
% \setcounter{subsection}{-1}

Passaremos a apresentar os experimentos requeridos.

% 2.1
\subsection{\textbf{TODO: Insert Section Title}}\label{sec:2.1}

\textbf{TODO}

% 2.2
\subsection{\textbf{TODO: Insert Section Title}}\label{sec:2.2}

\textbf{TODO}

% 2.3
\subsection{\textbf{TODO: Insert Section Title}}\label{sec:2.3}

\textbf{TODO}

% 2.4
\subsection{\textbf{TODO: Insert Section Title}}\label{sec:2.4}

\textbf{TODO}

\section{Execução de Programas no ZeptoProcessador-III}\label{sec:programs}

\subsection{Multiplicação de Números sem Sinal}\label{sec:programs:multu}

Processa a multiplicação dos números $5$ e $8$. Resultado esperado em R1 = 40.

\begin{table}[H]
    \centering
    \caption{Código para o programa \textbf{Multu}}
    \begin{tabular}{|l|l|l|l|}\hline
        \textbf{Endereço} & \textbf{Código Hexadecimal} & \textbf{Instrução} & \textbf{Comentário} \\\hline
        0x0000       & 0x0000 0010 & addi R1,R0,R0,0 & R1 = 0 ;; Resultado           \\\hline
        0x0001       & 0x0008 0020 & addi R2,R0,R0,8 & R2 = 8                        \\\hline
        0x0002       & 0x0005 0030 & addi R3,R0,R0,5 & R3 = 5                        \\\hline
        0x0003       & 0x0002 00FB & jal R15,Proc    & R15 = 0x0004; PC=Proc         \\\hline
        0x0004 Fim:  & 0x0000 1105 & beq R1,R1,0     & j Fim ;; mostra R1            \\\hline
        0x0005 Proc: & 0x0000 0040 & addi R4,R0,R0,0 & R4 = 0 ;; instantiate counter \\\hline
        0x0006 Loop: & 0x0000 2110 & addi R1,R1,R2,0 & R1 += R2                      \\\hline
        0x0007       & 0x0001 0440 & addi R4,R4,R0,1 & R4 += 1                       \\\hline
        0x0008       & 0xFFFE 3406 & bne  R4,R3,Loop & if (R4 \!= R3) goto Loop      \\\hline
        0x0009       & 0x0000 0F0C & jalr R0,R15,0   & Retorna resultado em R1       \\\hline
    \end{tabular}\label{tab:programs:multu}
\end{table}

\subsection{Multiplicação de Números com Sinal}\label{sec:programs:mult}

\begin{table}[H]
    \centering
    \caption{Código para o programa \textbf{Mult}}
    \begin{tabular}{|l|l|l|l|}\hline
        \textbf{Endereço} & \textbf{Código Hexadecimal} & \textbf{Instrução} & \textbf{Comentário}                 \\\hline
        0x0000        & 0x0000 0010 & addi R1,R0,R0,0      & R1 = 0 ;; Resultado                                   \\\hline
        0x0001        & 0x000X 0020 & addi R2,R0,R0,X      & R2 = X                                                \\\hline
        0x0002        & 0x000Y 0030 & addi R3,R0,R0,Y      & R3 = Y                                                \\\hline
        0x0003        & 0x0002 00FB & jal  R15,Proc        & R15 = 0x0004; PC=Proc                                 \\\hline
        0x0004 Fim:   & 0x0000 1105 & beq  R1,R1,0         & j Fim ;; mostra R1                                    \\\hline
        0x0005 Proc:  & 0x0000 0040 & addi R4,R0,R0,0      & R4 = 0 ;; counter (how many negative numbers?)        \\\hline
        0x0006        & 0x8000 0050 & addi R5,R0,R0,0x8000 & R5 is a mask to find if number is ODD or EVEN         \\\hline
        0x0007        & 0xFFFF 5262 & andi R6,R2,R5,0xFFFF & if (R6 == 0x08000) ODD else EVEN                      \\\hline
        0x0008        & 0x0004 5606 & bne  R6,R5,R2end     & R6 is a result of a mask; if (R6 == R5) ODD else EVEN \\\hline
        0x0009        & 0x0001 4040 & addi R4,R0,R4,1      & Increment count of odd numbers                        \\\hline
        0x000A        & 0xFFFF 0224 & xori R2,R2,R0,0xFFFF & One's compelemnt for R2                               \\\hline
        0x000B        & 0x0001 0220 & addi R2,R2,R0,1      & Two's compelemnt for R2                               \\\hline
        0x000C R2end: & 0xFFFF 5362 & andi R6,R3,R5,0xFFFF & if (R6 == 0x08000) ODD else EVEN                      \\\hline
        0x000D        & 0x0004 5606 & bne  R6,R5,R3end     & R6 is a result of a mask; if (R6 == R5) ODD else EVEN \\\hline
        0x000E        & 0x0001 0440 & addi R4,R4,R0,1      & Increment count of odd numbers                        \\\hline
        0x000F        & 0xFFFF 0334 & xori R3,R3,R0,0xFFFF & One's compelemnt for R3                               \\\hline
        0x0010        & 0x0001 0330 & addi R3,R3,R0,1      & Two's compelemnt for R3                               \\\hline
        0x0011 R3end: & 0x0000 0040 & addi R4,R0,R0,0      & R4 = 0 ;; initialize loop counter                     \\\hline
        0x0012 Loop:  & 0x0000 2110 & addi R1,R1,R2,0      & R1 += R2                                              \\\hline
        0x0013        & 0x0001 0440 & addi R4,R4,R0,1      & R4 += 1                                               \\\hline
        0x0014        & 0xFFFE 3406 & bne  R4,R3,Loop      & if (R4 \!= R3) goto Loop                              \\\hline
        0xWWWW        & 0x0001 0060 & addi R6,R0,R0,1      & R6 = 1                                                \\\hline
        0xWWWW        & 0x0003 6406 & bne  R4,R6,Ok        & if (R4 \!= R6) goto Ok                                \\\hline
        0xWWWW        & 0xFFFF 0114 & xori R1,R1,R0,0xFFFF & One's complement for R1                               \\\hline
        0xWWWW        & 0x0001 0110 & addi R1,R1,R0,1      & Two's complement for R1                               \\\hline
        0xWWWW Ok:    & 0x0000 0F0C & jalr R0,R15,0        & Retorna resultado em R1                               \\\hline
    \end{tabular}\label{tab:programs:mult}
\end{table}


\section{Análise dos Resultados}\label{sec:resultados}

Passaremos a analisar individualmente cada um dos tópicos anteriores, levantando
observações pertinentes para cada um deles.

\subsection{Análise do tópico~\ref{sec:2.1}}\label{sec:analise2.1}

\textbf{TODO}

\subsection{Análise do tópico~\ref{sec:2.2}}\label{sec:analise2.2}

\textbf{TODO}

\subsection{Análise do tópico~\ref{sec:2.3}}\label{sec:analise2.3}

\textbf{TODO}

\subsection{Análise do tópico~\ref{sec:2.4}}\label{sec:analise2.4}

\textbf{TODO}

\section{Conclusão}\label{sec:Conclusao}

\textbf{TODO}

\bibliographystyle{sbc}
\bibliography{relatorio}  %Aqui é a definição do arquivo .bib a ser usado pelas referências

\end{document}
