%%%%%%%%%%%%%%%%%%%%%%%%%%%%%%%%%%%%%%%%%%%%%%%%%%%%%%%%
% Este é um documento que servirá de modelo para
% os relatórios feitos na disciplina Laboratório de Circuitos Lógicos
% 2020-2
%%%%%%%%%%%%%%%%%%%%%%%%%%%%%%%%%%%%%%%%%%%%%%%%%%%%%%%%%

%%%%%%%%%%%%%%%%%%%%%%%%%%%%%%%%%%%%%%%%%%%%%%%%%%%%%%%%%
% Use os diferentes diretórios para colocar os relatórios de cada experimento, deste modo vc consegue manter um histórico e todo material organizado em apenas um local.
% Lembre-se de mudar o Main Document no Menu!!!

\documentclass[12pt]{article}

\usepackage{sbc-template}
\usepackage[brazil,american]{babel}
\usepackage[utf8]{inputenc}

\usepackage{graphicx}
\usepackage{url}
\usepackage{float}
\usepackage{listings}
\usepackage{color}
\usepackage{todonotes}
\usepackage{algorithmic}
\usepackage{algorithm}
\usepackage{hyperref}
\usepackage{amsmath}
\usepackage{graphicx}
\usepackage{array}
\usepackage{mwe}
\usepackage[shortlabels]{enumitem}

\usepackage{xcolor}
\usepackage{listings}
\usepackage[electronic]{ifsym}
\definecolor{vgreen}{RGB}{104,180,104}
\definecolor{vblue}{RGB}{49,49,255}
\definecolor{vorange}{RGB}{255,143,102}

\lstdefinestyle{verilog-style}
{
    language=Verilog,
    basicstyle=\small\ttfamily,
    keywordstyle=\color{vblue},
    identifierstyle=\color{black},
    commentstyle=\color{vgreen},
    numbers=left,
    numberstyle=\tiny\color{black},
    numbersep=10pt,
    tabsize=8,
    moredelim=*[s][\colorIndex]{[}{]},
    literate=*{:}{:}1
}

\makeatletter
\newcommand*\@lbracket{[}
\newcommand*\@rbracket{]}
\newcommand*\@colon{:}
\newcommand*\colorIndex{%
    \edef\@temp{\the\lst@token}%
    \ifx\@temp\@lbracket \color{black}%
    \else\ifx\@temp\@rbracket \color{black}%
    \else\ifx\@temp\@colon \color{black}%
    \else \color{vorange}%
    \fi\fi\fi
}
\makeatother

\usepackage{trace}

\sloppy


\title{Projeto Final\\
ZeptoProcessador-III de 16 Bits}

\author{Matheus Cardoso de Souza, 202033507\\
        Ualiton Ventura da Silva, 202033580\\
        Grupo G42
}

%%%% LEMBRE-SE DE MUDAR O GRUPO NA LINHA ABAIXO!!!!! %%%%%%
\address{Dep. Ciência da Computação -- Universidade de Brasília (UnB)\\
  CIC0231 - Laboratório de Circuitos Lógicos
  \email{matheus-cardoso.mc@aluno.unb.br, 202033580@aluno.unb.br}
}

\begin{document}
\maketitle

\selectlanguage{american}
 \begin{abstract}
   \textbf{TODO}
 \end{abstract}

\selectlanguage{brazil}
 \begin{resumo}
   \textbf{TODO}
 \end{resumo}


\section{Introdução}\label{sec:Introducao}

\textbf{TODO}

\subsection{Objetivos}\label{sec:Objetivos}

\textbf{TODO}

\subsection{Materiais}\label{sec:Materiais}

Em função da natureza do ensino a distância, os presentes experimentos não foram
realizados usando-se materiais e equipamentos físicos, mas sim emulados por meio
do software
\href{https://www.digitalelectronicsdeeds.com/downloads.html}{Deeds}.

A seguir estão enumerados os materiais utilizados:
\begin{itemize}
    \item Software Deeds
    \item Portas lógicas
    \begin{itemize}
      \item \emph{NAND}s de $2$ entradas
      \item \emph{NOR}s de $2$ entradas
      \item \emph{Flip-flops JK-net}s
      \item \emph{Display de saída de 8 Segmentos}
      \item \emph{Display de saída de 1 bit}
    \end{itemize}
    \item \emph{Clocks}
\end{itemize}

\section{Procedimentos}\label{sec:Procedimentos}
% \setcounter{subsection}{-1}

Passaremos a apresentar os experimentos requeridos.

% 2.1
\subsection{\textbf{TODO: Insert Section Title}}\label{sec:2.1}

\textbf{TODO}

% 2.2
\subsection{\textbf{TODO: Insert Section Title}}\label{sec:2.2}

\textbf{TODO}

% 2.3
\subsection{\textbf{TODO: Insert Section Title}}\label{sec:2.3}

\textbf{TODO}

% 2.4
\subsection{\textbf{TODO: Insert Section Title}}\label{sec:2.4}

\textbf{TODO}

\section{Análise dos Resultados}\label{sec:resultados}

Passaremos a analisar individualmente cada um dos tópicos anteriores, levantando
observações pertinentes para cada um deles.

\subsection{Análise do tópico~\ref{sec:2.1}}\label{sec:analise2.1}

\textbf{TODO}

\subsection{Análise do tópico~\ref{sec:2.2}}\label{sec:analise2.2}

\textbf{TODO}

\subsection{Análise do tópico~\ref{sec:2.3}}\label{sec:analise2.3}

\textbf{TODO}

\subsection{Análise do tópico~\ref{sec:2.4}}\label{sec:analise2.4}

\textbf{TODO}

\section{Conclusão}\label{sec:Conclusao}

\textbf{TODO}

\bibliographystyle{sbc}
\bibliography{relatorio}  %Aqui é a definição do arquivo .bib a ser usado pelas referências

\end{document}
